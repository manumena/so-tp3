\section{Pruebas}

Para compilar las pruebas hay que correr $make test$ parado en la carpeta $src$.
Luego para ejecutar las pruebas "mpiexec -np N ./test", donde N es la cantidad
de procesos a lanzar.

\subsection{Load}

Se ejecuta a $load$ con una lista de nombres de archivos [testfile1,testfile2,
testfile3,testfile4,testfile5]. Para probar que se cargaron las palabras
contenidas en los archivos, se recorre los archivos y por cada palabra se
ejecuta un $member$.

\subsection{Member}

Se ejecuta un $member$ pasandole como parametro "key", deberia responder que
no existe. Luego de ejecuta $load$ con una lista de nombres de archivos, se
recorre los archivos y por cada palabra se ejecuta un $member$, deberian
responder que existen. Despues se ejecuta $member$ pasandole como parametro
"key", deberia decir que no existe, ya que "key" no estaba en los archivos,
despues se ejecuta un $addAndInc$ pasandole como parametro "key", por último
se vuelve a ejecutar $member$ pasandole como parametro "key", deberia decir
que existe.

\subsection{Quit}

Al final del test se ejecuta un $quit$, se deberia ver como cada nodo escribe
quit antes de terminar.

\subsection{AddAndInc}

Se ejecuta $AddAndInc$ pasandole como parametro "key", luego se ejecuta un
$member$ pasandole como parametro "key", deberia decir que existe, previo al
$addAndInc$ se ejecuto $member$ pasandole como parametro "key" y respondio que
no existia, así probamos que $AddAndInc$ agrega la key.

Se ejecuta $AddAndInc$, pasandole como parametro "key", cinco veces, despues se
ejecuta $maximum$, deberia decir que el $string$ que más aparece es "key" con 6
apariciones, así probamos que la cantidad de aparariciones de la key "key" es
la misma cantidad de veces que se ejecuto $AddAndInc$.

\subsection{Maximum}

Luego de ejecutar $load$ pasandole como parametro una lista de nombres de
archivos, se ejecuta $maximum$, despues se ejecuta sies veces $AddAndInc$,
pasandole como parametro "key" por último se ejecuta $maximum$, este deberia
responder que el $string$ que más aparece es "key" con 6 apariciones, así
probamos que maximum responde la key que tiene más apariciones
