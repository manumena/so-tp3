\section{Pruebas}

Para compilar las pruebas hay que correr $make test$ parado en la carpeta $src$.
Luego para ejecutar las pruebas "mpiexec -np N ./test", donde N es la cantidad
de procesos a lanzar.

\subsection{Load}

Se ejecuta a $load$ con una lista de nombres de archivos [testfile1,testfile2,
testfile3,testfile4,testfile5]. Para probar que se cargaron las palabras
contenidas en los archivos, se recorre los archivos y por cada palabra se
ejecuta un $member$, estos deberian responder que existen.

\subsection{Member}

Se ejecuta un $member$ pasandole como parametro "key", deberia responder que
no existe. Luego de ejecuta $load$ con una lista de nombres de archivos, se
recorre los archivos y por cada palabra se ejecuta un $member$, deberian
responder que existen. Despues se ejecuta $member$ pasandole como parametro
"key", deberia decir que no existe, ya que "key" no estaba en los archivos,
despues se ejecuta un $addAndInc$ pasandole como parametro "key", por último
se vuelve a ejecutar $member$ pasandole como parametro "key", deberia decir
que existe.

\subsection{Quit}

Al final del test se ejecuta un $quit$, se deberia ver como cada nodo escribe
quit antes de terminar.

\subsection{AddAndInc}
Se ejecuta $member$ pasandole como parametro "key" y responde que
no existe, Luego se ejecuta $AddAndInc$ pasandole como parametro "key", luego
se ejecuta un $member$ pasandole como parametro "key", deberia decir que
existe, así probamos que $addAndInc$ agrega keys.

Se ejecuta dos veces $AddAndInc$, pasandole como parametro "key", despues se
ejecuta $maximum$, deberia decir que el $string$ que más aparece es "key" con 3
apariciones, así probamos que la cantidad de aparariciones de la key "key" es
la misma cantidad de veces que se ejecuto $AddAndInc$, entonces si existe la
key aumenta su cantidad de apariciones

\subsection{Maximum}

Luego de ejecutar $load$ pasandole como parametro una lista de nombres de
archivos, se ejecuta $maximum$, debera responder el string "palabraA" con dos
ocurrencias , despues se ejecuta 3 veces $addAndInc$, pasandole como parametro
"key", luego ejecuta $maximum$, se espera que responda que el $string$ que más
aparece es "key" con 3 apariciones, por último se ejecuta dos veces $addAndInc$
pasandole como parametro "palabraA" y se ejecuta $maximum$, debera responder
que el $string$  "palabraA"e s el maximo con 4 ocurrencias, así probamos que
$maximum$, retorna la key con más aceptaciones. 
