\section{Introducción}

El motivo de este trabajo práctico fue reforzar los conocimientos sobre
sistemas distribuídos mediante el envío de mensajes. Se realizó utilizando la
interfaz MPI, ejecutando varios procesos de los cuales uno (el nodo consola)
dirige al resto (indistinguibles entre sí) y se encarga de la entrada y salida
de los comandos.

El objetivo es desarrolar un mapa de hash distribuído y replicar el comportamiento
de ConcurrentHashMap pero utilizando distintos nodos (simulan ser máquinas
distintas) en lugar de tan solo hilos y que la comunicación sea entre máquinas
distintas mediante mensajes.

Antes de enviar cualquier mensaje, los nodos deben realizar tareas de
procesamiento durante una cantidad de tiempo no determinística. Este
comportamiento es simulado realizando llamadas a la función
\textbf{trabajarArduamente()} donde corresponda.

Por último, se incluye un conjunto de casos de prueba diseñados para verificar
los métodos y abordar todos los casos posibles, acompañado de una explicación
de cada uno y de sus objetivos.