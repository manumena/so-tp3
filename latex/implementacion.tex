\section{Implementación}

\subsection{Nodo}

El nodo tiene en su implementación un ciclo principal en donde en principio
se recibe un mensaje y se determina mediante \textit{probe} qué tipo de
mensaje es para seguir adelante con la operación correspondiente.

Cada uno posee su propio hashmap a modo de formar entre todos los nodos
un mapa distribuído.

\subsection{Load}

Este método recibe una lista de rutas a archivos y se encarga de cargarlos.

Esta implementación recorre la lista, y por cada archivo envía, de manera
no bloqueante, un mensaje a cada nodo con el nombre del archivo, solicitando
que el mismo sea cargado. Luego, se queda a la espera de una contestación. El
primer nodo en contestar el mensaje es el elegido para cargar el archivo; en
ese momento se envía un mensaje al elegido, confirmando la orden, y un mensaje
al resto comunicando que el archivo ya ha sido tomado por otro nodo. Una vez
hecho esto, se pasa a la próxima iteración y se realiza lo mismo para el
siguiente archivo.

Habiendo sido enviadas todas las órdenes de carga, el método se mantiene a la
espera de que los nodos contesten con un mensaje indicando que el archivo ha
sido cargado. El motivo de esto es que el tiempo que emplean los nodos para
cargar el archivo no es conocido por la consola, por lo que si no se esperase
una confirmación, el usuario podría intentar ejecutar otra operación mientras
que los archivos no han sido cargados por completo, pudiendo generar
inconsistencia.

\subsection{Member}

Recibe la clave que desea buscarse en el mapa distribuído. Envía de manera no
bloqueante la solicitud de búsqueda de la clave a todos los nodos y permanece
a la espera de la respuesta de cada uno, efectuando una recepción bloqueante
de mensajes. Al recibir las respuestas solo resta comprobar si la clave existe
en el mapa del nodo emisor, de ser falso, se continúa la recepción de
mensajes, de ser verdadero se establece la variable de retorno en verdadero y
se continúa la recepción para evitar que las respuestas perduren sin ser
leídas.

Por su parte, al recibir la orden de ejecutar \textit{member}, los nodos
solamente deben llamar a \textit{member} dentro del mapa que almacenan
localmente y enviar el resultado al nodo raíz.

\subsection{Quit}

Se envía de manera no bloqueante la señal de finalización con la cual cada
nodo rompe el ciclo principal y concluye su ejecución.

\subsection{AddAndInc}

Este método recibe una string key y se encarga de agregarlo si no existe ó
aumentar la cantidad de apareciones en el HashMap de algun nodo.

Esta implementación envía, de manera no bloqueante, un mensaje a cada nodo con
la key, solicitando que la misma se agrege. Luego se queda a la espera de una
constestación.  El primer nodo en contestar el mensaje es el elegido para
cargar la key; en ese momento se envía un mensaje al elegido , confirmando
la orden, y un mensaje al resto comunicando que la key ya fue agregada por
otro  nodo.

Habiendo sido enviadas todas las órdenes de carga, el método se mantiene a la
espera de que los nodos contesten con un mensaje indicando que la key haya
sido agregada. El motivo de esto es que el tiempo que emplean los nodos para
cargar el archivo no es conocido por la consola, por lo que si no se esperase
una confirmación, el usuario podría intentar ejecutar otra operación mientras
no se haya terminado de agregar la key, pudiendo generar inconsistencia.

\subsection{Maximum}

Se crea un hashmap para cargar todas las palabras presentes en los mapas de 
los nodos. Se le envía una señal a todos los nodos, indicando que la consola espera  
recibir todas las palabras. De manera bloqueante se reciben las palabras
y se cargan en el mapa creado, hasta que los nodos indiquen que enviaron todas sus 
palabras mediante enviar una referencia a 'NULL'. Por último, se calcula el máximo 
utilizando la función maximum() de la clase HashMap.

Los nodos registran el pedido de envío de todas sus palabras y realizan la función 
trabajarArduamente() para hacer tiempo entre que se recibe el mensaje hasta que
se comienzan a enviar las palabras. Luego procede a enviarlas de manera no bloqueante 
con la funcion MPI_Isend(), de modo que la consola las obtenga cuando este disponible.
Una vez que termina se vuelve a realizar la función trabajarArduamente() para separar
el envío de las palabras validas del caracter 'NULL' que indica que no se enviarán mas palabras.
