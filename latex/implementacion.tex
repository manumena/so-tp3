\section{Implementación}

\subsection{Nodo}

El nodo tiene en su implementación un ciclo principal en donde en principio
se recibe un mensaje y se determina mediante \textit{probe} qué tipo de
mensaje es para seguir adelante con la operación correspondiente.

Cada uno posee su propio hashmap a modo de formar entre todos los nodos
un mapa distribuído.

\subsection{Load}

Este método recibe una lista de rutas a archivos y se encarga de cargarlos.

Esta implementación recorre la lista, y por cada archivo envía, de manera
no bloqueante, un mensaje a cada nodo con el nombre del archivo, solicitando
que el mismo sea cargado. Luego, se queda a la espera de una contestación. El
primer nodo en contestar el mensaje es el elegido para cargar el archivo; en
ese momento se envía un mensaje al elegido, confirmando la orden, y un mensaje
al resto comunicando que el archivo ya ha sido tomado por otro nodo. Una vez
hecho esto, se pasa a la próxima iteración y se realiza lo mismo para el
siguiente archivo.

\subsection{Member}

Recibe la clave que desea buscarse en el mapa distribuído. Envía de manera no
bloqueante la solicitud de búsqueda de la clave a todos los nodos y permanece
a la espera de la respuesta de cada uno, efectuando una recepción bloqueante
de mensajes. Al recibir las respuestas solo resta comprobar si la clave existe
en el mapa del nodo emisor, de ser falso, se continúa la recepción de
mensajes, de ser verdadero se establece la variable de retorno en verdadero y
se continúa la recepción para evitar que las respuestas perduren sin ser
leídas.

Por su parte, al recibir la orden de ejecutar \textit{member}, los nodos
solamente deben llamar a \textit{member} dentro del mapa que almacenan
localmente y enviar el resultado al nodo raíz.

\subsection{Quit}

Se envía de manera no bloqueante la señal de finalización con la cual cada
nodo rompe el ciclo principal y concluye su ejecución.
